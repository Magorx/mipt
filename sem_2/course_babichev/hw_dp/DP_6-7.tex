\documentclass{article}
\usepackage[T2A]{fontenc}
\usepackage[utf8x]{inputenc}
\usepackage[russian]{babel}
\usepackage{amsmath}
\usepackage{amstext}
\textheight=30cm
\textwidth=17cm
\oddsidemargin=-30pt
\topmargin=-2cm
\parskip=5pt 
\begin{document}

{\bf
Домашнее задание \textnumero 6 - 7.

Савчук Анна}

{\bf Задача 1.}

ДП будем строить размера $1 * 2^{n}$, т.е. по маскам. 
dp\[mask\] будет хранить уже принесённые предметы, изначально заполним его очень большими числами (большими, чем они могли бы быть, условно $\infty$).
Далее вложенными циклами от 0 до n (индексы i, j) будем делать следующее:
1) Если iго и jго индекса в текущей маске нет, то посчитаем минимальное расстояние до них и обновим dp\[new_mask\] = min(dp\[mask\] + dist, dp\[new_mask\]), где $new_mask = mask + 2^{i} + 2^{j}$. 
2) В конце возьмём значение из элемента $dp(2^{n})$.

Чтобы запомнить последовательность действий, будем дополнительно хранить массив parent, в котором для каждого parent\[mask\] будем запоминать, откуда мы изменили dp\[mask\] (уменьшили его). Далее циклом от dp\[mask\] будем идти в сторону уменьшения parent[mask].

{\bf Задача 2.}

Будем  решать задачу о рюкзаке точно также, только разобъём решение на несколько по $sqrt{n}$ предметов. Тогда, изначально насчитав ДП для первых $sqrt{n}$ предметов, мы возьмём следующие, а конечное ДП для предыдущих сохраним, т.е. будем хранить по W строк размера  $sqrt{n}$ для пересчёта и ещё одну текущую. В них также будем хранить ячейку, откуда мы пришли.